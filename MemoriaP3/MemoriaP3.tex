\documentclass[twoside]{article}


% ------
% Fonts and typesetting settings
%\usepackage[sc]{mathpazo}
\usepackage[T1]{fontenc}
\linespread{1.05} % Palatino needs more space between lines
\usepackage{microtype}


% ------
% Page layout
\usepackage[hmarginratio=1:1,top=32mm,columnsep=20pt]{geometry}
\usepackage[font=it]{caption}
\usepackage{paralist}
\usepackage{multicol}

% ------
% Lettrines
\usepackage{lettrine}


% ------
% Abstract
\usepackage{abstract}
	\renewcommand{\abstractnamefont}{\normalfont\bfseries}
	\renewcommand{\abstracttextfont}{\normalfont\small\itshape}


% ------
% Titling (section/subsection)
\usepackage{titlesec}
\renewcommand\thesection{\Roman{section}}
\titleformat{\section}[block]{\large\scshape\centering}{\thesection.}{1em}{}


% ------
% Header/footer
\usepackage{fancyhdr}
	\pagestyle{fancy}
	\fancyhead{}
	\fancyfoot{}
	\fancyhead[C]{Metaheurísticas $\bullet$ \date{\today} $\bullet$ Práctica 3 - Genéticos}
	\fancyfoot[RO,LE]{\thepage}


% ------
% Clickable URLs (optional)
\usepackage{hyperref}

% ------
% Maketitle metadata
\title{\vspace{-15mm}%
	\fontsize{24pt}{10pt}\selectfont
	\textbf{Práctica 3.a: Algoritmos Genéticos para el Problema de la
	Asignación Cuadrática}
	}
\author{%
	\large
	\textsc{Alejandro Alcalde}\thanks{Template by \href{http://www.howtotex.com}{howtoTeX.com}, \href{http://www.elbauldelprogramador.com}{elbauldelprogramador.com}} \\[2mm]
	\normalsize	Tercer curso Grado en Ingeniería Informática, Granada, grupo de los miércoles a las 17.30 \\
	\normalsize	\href{mailto:algui91@gmail.com}{algui91@gmail.com}	\\
	\normalsize Algoritmos: Generacional con pos y PMX, Estacionario con pos y PMX
	\vspace{-5mm}
	}
\date{}

%%%%%%%%%%%%%%%%%%%%%%%%%%%%%%%%%%%%%%%%%%%%%%%%%%%%%%%%%%%%%%%%%%%%%%%%
% MIS CAMBIOS
%%%%%%%%%%%%%%%%%%%%%%%%%%%%%%%%%%%%%%%%%%%%%%%%%%%%%%%%%%%%%%%%%%%%%%%%

\input{"/home/hkr/Drive/Grado_Ing_Informatica/Global/myConfig"}
\usepackage{booktabs} % Para la tabla
%\let\oldtabular\tabular
%\renewcommand{\tabular}{\footnotesize\oldtabular}
\usepackage{graphicx}
\usepackage{amsmath}                                    % Math



\begin{document}

\maketitle
\thispagestyle{fancy}
\tableofcontents
\newpage

\begin{abstract}
\noindent Memoria de la tercera práctica de la asignatura metaheurísticas
del tercer curso del Grado en Ingeniería Informatica de la facultad de Granada.
La practica consiste en la implementación de dos tipos de algoritmos genéticos,
uno generacional y otro elitista. Además, se han implementado dos tipos
de operadores de cruce, PMX y basado en posición, lo que hace un total de
cuatro implementaciones de algoritmos genéticos.
\end{abstract}

%\begin{multicols}{2}

\section{Descripción del problema}

\lettrine[nindent=0em,lines=3]{E}l problema de la asignación cuadrática (QAP) es un problema estándar en teoría de localización. En éste se trata de asignar N unidades a una cantidad N de sitios o localizaciones en donde se considera un costo asociado a cada una de las asignaciones. Este costo dependerá de las distancias y flujo entre las unidades, además de un costo adicional por asignar cierta unidad a cierta localización específica. De este modo se buscará que este costo, en función de la distancia y flujo, sea mínimo.

\subsection{Definición matemática}

\begin{displaymath}
_{S\in\prod _N}^{min}\left ( \sum_{i=1}^n \sum_{j=1}^n f_{ij} \cdot d_{S(i)S(j)}  \right )
\end{displaymath}

Donde $\prod _N$ es el conjunto de todas las permutaciones posibles de $N={1,2,\dots,n}$

\subsection{Representación del problema}

En éste problema las soluciones se pueden representar como permutaciones de un conjunto. Si el problema es de tamaño cuatro, por ejemplo, una solución vendría dada por la permutación $N=\{3,2,1,4\}$. Si tomamos los índices de éste conjunto como las unidades, y el valor en dicho índice como localizaciones, la localización 3 estaría asignada a la unidad 1, la localización 2 a la unidad 2 etc.

El objetivo del problema es \textbf{minimizar} la expresión mostrada anteriormente. Ésta será la función objetivo:

\begin{displaymath}
_{S\in\prod _N}^{min}\left ( \sum_{i=1}^n \sum_{j=1}^n f_{ij} \cdot d_{S(i)S(j)}  \right )
\end{displaymath}

Hará falta un mecanismo para \textbf{generar la solución inicial}, en nuestro caso, la permutación $N$ inicial sobre la que lanzar los distintos algoritmos será aleatoria.

De igual modo será necesario un esquema de \textbf{generación de soluciones vecinas}. Para ello, se realizará un intercambio entre dos localizaciones, cambiando sus respectivas unidades por la otra.

\section{Composición de los algoritmos}

\subsection{Esquema de representación}

Como se ha mencionado anteriormente, el esquema elegido para representar
una solución ha sido una permutación del tipo $\{1,2,3,4\}$ en la que los
elementos representan localizaciones y la posición que ocupan las unidades
a las que han sido asignadas.

\subsection{Descripción de la función objetivo}

La función objetivo descrita con su fórmula en apartados anteriores puede
representarse con el siguiente pseudocódigo:

\begin{pythoncode}
i = 0 .. Tamaño del problema
    j = 0 .. Tamaño del problema
        acumular el coste producido al asociar el flujo existente entre
        la unidad i y j y la distancia existente entre las localizaciones
        i y j
\end{pythoncode}

\subsection{Generación de soluciones aleatorias}

El tamaño de la población para esta práctica debe ser de 50, por lo tanto
es necesario generar 50 soluciones iniciales aleatorias. Para así disponer
de una primera población. El mecanismo seguido para obtener la población
inicial, dada una semilla inicial, es el siguiente:

\begin{pythoncode}
desde i=0 hasta el tamaño de la población
    solucion aleatoria = []
    mientras el tamaño de la solución no sea correcto
        generar un número aleatorio entre 0 y el tamaño de la solución
            si el número ya está en la solución descartarlo y generar uno nuevo
    añadir la solución aleatoria a la población
\end{pythoncode}

Éste proceso se realiza al principio de la ejecución del algoritmo.

\subsection{Mecanismo de selección}

El mecanismo de selección escogido ha sido el torneo binario, éste método
consiste en escoger aleatoriamente dos individuos de la población, quedarse
con el mejor de ambos y luego volver a coger otros dos y repetir el proceso.
Los ganadores de los torneos formarán parte de la población en la siguiente
generación. La diferencia entre el esquema generacional y el estacionario reside en que el primero realiza tantos
torneos como individuos existan en la población, mientras que el último
realiza dos veces el torneo para seleccionar únicamente dos individuos.

Para la versión generacional la selección se ha implementado de la siguiente forma:

\begin{pythoncode}
desde i = 0 hasta el tamaño de la población
    seleccionar aleatoriamente dos individuos de la población distintos
    añadir a la siguiente generación el mejor individuo de los dos
\end{pythoncode}

La versión estacionaria es un caso particular de la anterior en la que se
realiza el torneo dos veces.

\subsection{Operador de cruce}

Una vez se ha obtenido una nueva población mediante la selección anterior es necesario
realizar un cruce entre sus miembros bajo una cierta probabilidad. Para evitar
la generación de múltiples números aleatorios se ha optado por usar la esperanza
matemática para calcular de antemano cuantos individuos cruzarán.

El cruce consiste en combinar dos padres para obtener dos hijos que preserven
de alguna manera la calidad de ambos padres. Se han considerado dos tipos de
operadores, basado en posición y PMX. La descripción de ambos se muestra a continuación:

\begin{pythoncode}
# Basado en posición
desde i = 0 hasta esperanza matemática del cruce
    - Se seleccionan dos padres consecutivos de la población
    - Se comprueba qué elementos coinciden en los dos padres y se añaden al hijo1 y hijo2
    - Los elementos que no coinciden se desordenan y se añaden al hijo1
    - Se vuelven a desordenar los elementos que no coindicen y se añaden al hijo2
    - Se sustituye a los dos padres seleccionados con los dos hijos
    - Marcar los dos nuevos individuos para su posterior evaluación
\end{pythoncode}

\begin{pythoncode}
# PMX
desde i = 0 hasta esperanza matemática del cruce
    - Se seleccionan dos padres consecutivos de la población
    - Generar dos puntos de corte aleatorios, c1 entre 0 y n, y c2 entre c1 y n
    - Copiar en el hijo1 los elementos entre los puntos de corte del padre2
    - Copiar en el hijo2 los elementos entre los puntos de corte del padre1
    - Generar las correspondencias existentes de los elementos de ambos padres
      entre los puntos de corte
    - Obtener una lista de índices sin elementos asignados en los hijos
    - Obtener qué elementos ya existen en el hijo1 y hijo2
    - Recorrer los índices si rellenar en el hijo1
        - Si el elemento a insertar del padre1 ya existe en el hijo1
            - Sustituirlo por la correspondencia adecuada
        - Si no, agregar el elemento del padre1 al hijo1
    - Recorrer los índices si rellenar en el hijo2
        - Si el elemento a insertar del padre2 ya existe en el hijo2
            - Sustituirlo por la correspondencia adecuada
        - Si no, agregar el elemento del padre2 al hijo2
    - Sustituir a los dos padres seleccionados por los dos hijos
    - Marcar los dos nuevos individuos para su posterior evaluación
\end{pythoncode}

Hay una pequeña diferencia respecto al esquema generacional y el estacionario.
El generacional realiza tantos cruces como su esperanza matemática indique. El
estacionario debe cruzar con una probabilidad de 1, ya que únicamente selecciona dos padres.

\subsection{Operador de mutación}

La aplicación de la mutación también se basa en una probabilidad. Al igual
que en el cruce, para evitar la generación innecesaria de números aleatorios, se
calculan con antelación la cantidad exacta de elementos a mutar en base a esa probabilidad.
El operador de mutación es el mismo usado en prácticas anteriores, escoger dos posiciones
aleatorias e intercambiarlas.

En esta caso el procedimiento es el mismo para ambos esquemas:

\begin{pythoncode}
desde i = 0 hasta la esperanza matemática para la mutación
    - Generar un número aleatorio que seleccionará el elemento de la población a mutar
    - Generar dos números más para decidir qué genes (elementos) del individuo intercambiar
    - Aplicar el intercambio y marcar el individuo para su posterior re-evaluación
\end{pythoncode}

En el estacionario, la selección del individuo se realizará sobre dos elementos.

\subsection{Evaluación}

Para evitar realizar cálculos innecesarios, durante una generación se marcan
los individuos que necesitarán ser re-evaluados. Calculando únicamente el coste
de dichos individuos en el caso de estar marcados para re-evaluación. En
esta etapa se guarda el mejor individuo de la población.


\section{Estructura del método de búsqueda}

\subsection{Esquema de evolución y reemplazamiento generacional}

Éste esquema consiste en seleccionar una población de igual tamaño a la población
genética. En nuestro caso la población es de 50. Por tanto, en cada iteración (generación)
se guarda la población anterior como “población vieja” y se sustituye por una nueva
con el mecanismo de selección, también de tamaño 50. A continuación se muestra el esquema
de evolución:

\begin{pythoncode}
Inicializar población
Evaluar población

mientras no se cumpla el criterio de parada
    Intercambiar la población vieja por una nueva
    Seleccionar individuos de la población vieja para formar parte de la nueva
    Cruzar
    Mutar
    Reemplazar
    Evaluar
\end{pythoncode}

En el esquema de reemplazamiento se sustituye la generación vieja por la nueva.
Pero se pretende conservar el elitismo. Para ello, si el mejor individuo
de la población vieja no está en la nueva, se sustituirá la peor solución
de la población actual con éste individuo. El pseudocódigo es el siguiente:

\begin{pythoncode}
Si el mejor individuo de la población anterior no está en la nueva
    - Buscar la peor solución de la población actual y sustituirla por el
      individuo de la población anterior.
\end{pythoncode}

\subsection{Esquema de evolución y reemplazamiento estacionario}

En éste esquema se seleccionan únicamente dos padres, los cuales se
cruzarán, mutarán. El reemplazamiento se realizará únicamente en el caso
que se produzca una mejora, es decir, si algunos de los individuos generados
es mejor que los peores de la población.

\begin{pythoncode}
Inicializar población
Evaluar población

mientras no se cumpla el criterio de parada
    Seleccionar 2 individuos de la población
    Cruzarlos
    Mutar con probabilidad
    Reemplazar
    Evaluar
\end{pythoncode}

El reemplazamiento:

\begin{pythoncode}
Si el peor de la población es peor que el primer individuo generado
    - Sustituirlo
Si el segundo peor de la población es peor que el segundo individuo generado
    - Sustituirlo
Si el peor de la población es peor que el segundo individuo generado
    - Sustituirlo
Si el segundo peor de la población es peor que el segundo generado
    - Sustituirlo
\end{pythoncode}

\section{Algoritmo de comparación, Greedy}

Consiste simplemente en asociar unidades de gran flujo con localizaciones céntricas en la red y viceversa. Para ello se calcula el potencial de flujo y de distancia. A mayor potencial de flujo, más peso tendrá dicha unidad en el intercambio de flujos y, a menor flujo de distancia, más céntrica será la localización. Por tanto el algoritmo selecciona la unidad disponible con mayor potencial de flujo y le asignará la localización de menor potencial de distancia.

\begin{pythoncode}
    Calculo potenciales.
    Inicialzar N a 0.
    para i = 1 hasta el tamaño del problema
        Seleccionar unidad de mayor potencial de flujo
        Seleccionar localización de menor potencial de distancia
        Añadir la localización seleccionada al índice correspondiente a la unidad en N
\end{pythoncode}

\section{Procedimiento considerado}

La implementación se ha realizado en Python, basándose en las explicaciones
de clase y los documentos proporcionados por el profesorado. Para ejecutar el programa:

\begin{bashcode}
$ python QAP.py -d <datos del problema> -a \
[elitist_gg_pos|elitist_gg_pmx|stacionary_gg_pos|stacionary_gg_pmx] -s semilla -v<verbose>
\end{bashcode}

\section{Experimentos y análisis de resultados}

Para esta pŕactica se ha usado todos los casos, salvo los bur. La semilla
usada fue 2704647398. Haciendo pruebas se descubrió un valor cercano al óptimo para la instancia
nug25. El valor óptimo de ésta instancia es 3744, con la semilla 1411423505
se obtuvo un coste de 3764.

A continuación se muestra la tabla con los resultados para los distintos algoritmos.

\begin{table}[h]
\centering
    \begin{tabular}{lll}
    \toprule
    Algoritmo               & Desviación & Tiempo \\
    \midrule
    Greedy&81.4492028061&0.0012273788\\
    AGG-pos&10.425309152&18.3893676599\\
    AGG-PMX&10.1836523477&17.2611667977\\
    AGE-pos&44.6100486995&14.826227612\\
    AGE-PMX&43.8634597082&14.8372759422\\
    \bottomrule
    \end{tabular}
    \caption{}
\end{table}
%\end{multicols}


\begin{table}[h]
\centering
    \begin{tabular}{llll}
    \hline
    \multicolumn{4}{c}{Genético Generacional POS} \\
    \toprule
    Caso               & Coste & Desv & Tiempo \\
    \midrule
    Els19&17535402&1.88&10.4644160271\\
    Chr20a&2838&29.47&8.3315520287\\
    Chr25a&5942&56.53&12.3692309856\\
    Nug25&3904&4.27&16.6173238754\\
    Tai30a&1913302&5.23&15.8454630375\\
    Tai30b&709753517&11.40&15.0750370026\\
    Esc32a&160&23.08&10.9133908749\\
    Kra32&91770&3.46&12.6483390331\\
    Tai35a&2559336&5.67&15.4616510868\\
    Tai35b&305457393&7.82&14.1183140278\\
    Tho40&248844&3.46&16.4316048622\\
    Tai40a&3278940&4.45&16.9408519268\\
    Sko42&16410&3.78&16.6719579697\\
    Sko49&24372&4.22&20.7642681599\\
    Tai50a&5295072&7.21&21.1576180458\\
    Tai50b&486611071&6.06&21.464343071\\
    Tai60a&7815502&8.46&27.776927948\\
    Lipa90a&364988&1.21&57.9563279152\\
    \bottomrule
    \end{tabular}
    \caption{}
\end{table}


\begin{table}[h]
\centering
    \begin{tabular}{llll}
    \hline
    \multicolumn{4}{c}{Genético Generacional PMX} \\
    \toprule
    Caso               & Coste & Desv & Tiempo \\
    \midrule
    Els19&17937024&4.21&6.3874239922\\
    Chr20a&2896&32.12&7.4710178375\\
    Chr25a&6084&60.27&8.6566560268\\
    Nug25&3920&4.70&9.7244329453\\
    Tai30a&1889940&3.95&12.336566925\\
    Tai30b&691796918&8.58&12.4235701561\\
    Esc32a&156&20.00&10.8866541386\\
    Kra32&94510&6.55&11.614978075\\
    Tai35a&2518224&3.97&13.8358850479\\
    Tai35b&302709491&6.85&13.8314809799\\
    Tho40&253804&5.52&15.98488307\\
    Tai40a&3311914&5.50&15.962831974\\
    Sko42&16434&3.93&17.0838091373\\
    Sko49&23950&2.41&21.2035839558\\
    Tai50a&5181558&4.92&21.168197155\\
    Tai50b&469403774&2.31&21.6532700062\\
    Tai60a&7673216&6.48&28.1585299969\\
    Lipa90a&364356&1.03&62.3172309399\\
    \bottomrule
    \end{tabular}
    \caption{}
\end{table}


\begin{table}[h]
\centering
    \begin{tabular}{llll}
    \hline
    \multicolumn{4}{c}{Genético Estacionario POS} \\
    \toprule
    Caso               & Coste & Desv & Tiempo \\
    \midrule
    Els19&24385666&41.67&4.5428960323\\
    Chr20a&5414&146.99&4.8091681004\\
    Chr25a&11466&202.05&6.5609920025\\
    Nug25&4404&17.63&6.5850667954\\
    Tai30a&2072420&13.99&8.5201299191\\
    Tai30b&817177373&28.26&8.8779101372\\
    Esc32a&294&126.15&9.3563039303\\
    Kra32&118480&33.57&9.7311291695\\
    Tai35a&2773072&14.50&10.8184459209\\
    Tai35b&370803500&30.88&10.9540109634\\
    Tho40&302924&25.95&13.8538200855\\
    Tai40a&3578576&13.99&13.3199338913\\
    Sko42&18734&18.48&14.8727619648\\
    Sko49&27392&17.13&19.4602730274\\
    Tai50a&5666064&14.73&19.6195249557\\
    Tai50b&646546149&40.91&20.1130871773\\
    Tai60a&8246186&14.44&27.0534820557\\
    Lipa90a&366624&1.66&57.8231608868\\
    \bottomrule
    \end{tabular}
    \caption{}
\end{table}

\begin{table}[h]
\centering
    \begin{tabular}{llll}
    \hline
    \multicolumn{4}{c}{Genético Estacionario PMX} \\
    \toprule
    Caso               & Coste & Desv & Tiempo \\
    \midrule
    Els19&24094508&39.98&4.6574339867\\
    Chr20a&5386&145.71&4.9204189777\\
    Chr25a&10854&185.93&6.6915891171\\
    Nug25&4322&15.44&6.609251976\\
    Tai30a&2067496&13.71&8.4478189945\\
    Tai30b&870376734&36.61&8.62398386\\
    Esc32a&284&118.46&9.2911510468\\
    Kra32&118060&33.10&9.5825548172\\
    Tai35a&2783710&14.93&10.8890759945\\
    Tai35b&392475721&38.53&11.1784491539\\
    Tho40&303358&26.13&13.6819250584\\
    Tai40a&3606472&14.88&13.4146249294\\
    Sko42&18566&17.42&14.84770298\\
    Sko49&27258&16.56&19.39554286\\
    Tai50a&5659200&14.59&19.5065131187\\
    Tai50b&649205197&41.49&19.9771239758\\
    Tai60a&8246650&14.44&26.7352631092\\
    Lipa90a&366480&1.62&58.6205430031\\
    \bottomrule
    \end{tabular}
    \caption{}
\end{table}


Observando las tablas queda patente que la diferencia entre los operadores
de cruce no es muy significativa. El PMX es ligeramente mejor, probablemente
debido a que realiza cortes en las soluciones padres y las pega en los hijos,
lo que ayuda a conservar gran parte del valor de la solución del padre en el hijo.
El PMX intenta combinar los dos padres conservando en la medida de lo posible las
posiciones de ambos en sus hijos.

Una razón por la cual el operador basado en posición pueda actuar un poco peor
es que se basa únicamente en conservar las posiciones coincidentes en ambos padres, haciendo
así que la porción de solución que se conserva tienda a ser menor que en PMX.
El factor de aleatoriedad que se aplica en este operador para reasignar las
posiciones sobrandes puede proporcionar tanto soluciones malas como buenas. Por tanto
el PMX tiende a conservar más las propiedades de los padres.


En cuanto a los tos esquemas, generacional o estacionario, está claro que
el estacionario, por su forma se seleccionar población siempre va a explorar
menor cantidad en el espacio de soluciones, ya que por cada generación se
seleccionan únicamente dos elementos para cruzar y mutar, mientras que la
versión generacional explorará a un ritmo de 50 individuos por generación. Realizándole
además mutaciones y cruces a los individuos correspondientes.

\end{document}
