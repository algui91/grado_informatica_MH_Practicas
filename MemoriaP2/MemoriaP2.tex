\documentclass[twoside]{article}


% ------
% Fonts and typesetting settings
%\usepackage[sc]{mathpazo}
\usepackage[T1]{fontenc}
\linespread{1.05} % Palatino needs more space between lines
\usepackage{microtype}


% ------
% Page layout
\usepackage[hmarginratio=1:1,top=32mm,columnsep=20pt]{geometry}
\usepackage[font=it]{caption}
\usepackage{paralist}
\usepackage{multicol}

% ------
% Lettrines
\usepackage{lettrine}


% ------
% Abstract
\usepackage{abstract}
	\renewcommand{\abstractnamefont}{\normalfont\bfseries}
	\renewcommand{\abstracttextfont}{\normalfont\small\itshape}


% ------
% Titling (section/subsection)
\usepackage{titlesec}
\renewcommand\thesection{\Roman{section}}
\titleformat{\section}[block]{\large\scshape\centering}{\thesection.}{1em}{}


% ------
% Header/footer
\usepackage{fancyhdr}
	\pagestyle{fancy}
	\fancyhead{}
	\fancyfoot{}
	\fancyhead[C]{Metaheurísticas $\bullet$ \date{\today} $\bullet$ Práctica 3 - Genéticos}
	\fancyfoot[RO,LE]{\thepage}


% ------
% Clickable URLs (optional)
\usepackage{hyperref}

% ------
% Maketitle metadata
\title{\vspace{-15mm}%
	\fontsize{24pt}{10pt}\selectfont
	\textbf{Práctica 2.a Búsquedas Multiarranque para el Problema de la Asignación Cuadrática}
	}
\author{%
	\large
	\textsc{Alejandro Alcalde}\thanks{Template by \href{http://www.howtotex.com}{howtoTeX.com}, \href{http://www.elbauldelprogramador.com}{elbauldelprogramador.com}} \\[2mm]
	\normalsize	Tercer curso Grado en Ingeniería Informática, Granada, grupo de los miércoles a las 17.30 \\
	\normalsize	\href{mailto:algui91@gmail.com}{algui91@gmail.com}	\\
	\normalsize Algoritmos: Generacional con pos y PMX, Estacionario con pos y PMX
	\vspace{-5mm}
	}
\date{}

%%%%%%%%%%%%%%%%%%%%%%%%%%%%%%%%%%%%%%%%%%%%%%%%%%%%%%%%%%%%%%%%%%%%%%%%
% MIS CAMBIOS
%%%%%%%%%%%%%%%%%%%%%%%%%%%%%%%%%%%%%%%%%%%%%%%%%%%%%%%%%%%%%%%%%%%%%%%%

\input{"/home/hkr/Drive/algui91/Grado_Ing_Informatica/Global/myConfig"}
\usepackage{booktabs} % Para la tabla
%\let\oldtabular\tabular
%\renewcommand{\tabular}{\footnotesize\oldtabular}
\usepackage{graphicx}
\usepackage{amsmath}                                    % Math



\begin{document}

\maketitle
\thispagestyle{fancy}
\tableofcontents
\newpage

\begin{abstract}
\noindent Memoria de la tercera práctica de la asignatura metaheurísticas
del tercer curso del Grado en Ingeniería Informatica de la facultad de Granada.
La practica consiste en la implementación de dos tipos de algoritmos genéticos,
uno generacional y otro elitista. Además, se han implementado dos tipos
de operadores de cruce, PMX y basado en posición, lo que hace un total de
cuatro implementaciones de algoritmos genéticos.
\end{abstract}

%\begin{multicols}{2}

\section{Descripción del problema}

\lettrine[nindent=0em,lines=3]{E}l problema de la asignación cuadrática (QAP) es un problema estándar en teoría de localización. En éste se trata de asignar N unidades a una cantidad N de sitios o localizaciones en donde se considera un costo asociado a cada una de las asignaciones. Este costo dependerá de las distancias y flujo entre las unidades, además de un costo adicional por asignar cierta unidad a cierta localización específica. De este modo se buscará que este costo, en función de la distancia y flujo, sea mínimo.

\subsection{Definición matemática}

\begin{displaymath}
_{S\in\prod _N}^{min}\left ( \sum_{i=1}^n \sum_{j=1}^n f_{ij} \cdot d_{S(i)S(j)}  \right )
\end{displaymath}

Donde $\prod _N$ es el conjunto de todas las permutaciones posibles de $N={1,2,\dots,n}$

\subsection{Representación del problema}

En éste problema las soluciones se pueden representar como permutaciones de un conjunto. Si el problema es de tamaño cuatro, por ejemplo, una solución vendría dada por la permutación $N=\{3,2,1,4\}$. Si tomamos los índices de éste conjunto como las unidades, y el valor en dicho índice como localizaciones, la localización 3 estaría asignada a la unidad 1, la localización 2 a la unidad 2 etc.

El objetivo del problema es \textbf{minimizar} la expresión mostrada anteriormente. Ésta será la función objetivo:

\begin{displaymath}
_{S\in\prod _N}^{min}\left ( \sum_{i=1}^n \sum_{j=1}^n f_{ij} \cdot d_{S(i)S(j)}  \right )
\end{displaymath}

Hará falta un mecanismo para \textbf{generar la solución inicial}, en nuestro caso, la permutación $N$ inicial sobre la que lanzar los distintos algoritmos será aleatoria.

De igual modo será necesario un esquema de \textbf{generación de soluciones vecinas}. Para ello, se realizará un intercambio entre dos localizaciones, cambiando sus respectivas unidades por la otra.

\section{Composición de los algoritmos}

\subsection{Esquema de representación}

Como se ha mencionado anteriormente, el esquema elegido para representar
una solución ha sido una permutación del tipo $\{1,2,3,4\}$ en la que los
elementos representan localizaciones y la posición que ocupan las unidades
a las que han sido asignadas.

\subsection{Descripción de la función objetivo}

La función objetivo descrita con su fórmula en apartados anteriores puede
representarse con el siguiente pseudocódigo:

\begin{pythoncode}
i = 0 .. Tamaño del problema
    j = 0 .. Tamaño del problema
        acumular el coste producido al asociar el flujo existente entre
        la unidad i y j y la distancia existente entre las localizaciones
        i y j
\end{pythoncode}

\subsection{Generación de soluciones aleatorias}

El mecanismo de generación de soluciones aleatorias tanto para ILS
como para BMB ha sido el siguiente:

\begin{pythoncode}
solucion aleatoria = []
mientras el tamaño de la solución no sea correcto
    generar un número aleatorio entre 0 y el tamaño de la solución
        si el número ya está en la solución descartarlo y generar uno nuevo
\end{pythoncode}

\subsection{Descripción de la BL}

La búsqueda local usada utiliza el mecanismo Dont Look Bits para ahorrar tiempo
de exploración innecesario. Se realizan un total de 10000 evaluaciones:

\begin{pythoncode}
mientras no se cumpla el criterio de parada
    recorrer la DLB
        Si el bit está a 0
            Explorar el entorno
                Realizar un intercambio
                Si se mejora aplicar el intercambio
            Si se explora todo el entorno poner el bit de la DLB a 1
\end{pythoncode}

El operador de vecino simplemente intercambia una posición de la solución
con otra, y luego se aplica la siguiente función de factorización para ahorrar
el cálculo de la función objetivo:

\begin{displaymath}
    \sum_{k \neq r,s}
\begin{bmatrix}
 f_{rk}\cdot (d_{\pi (s)\pi(k)} -  d_{\pi (r)\pi(k)})
+ f_{sk}\cdot (d_{\pi (r)\pi(k)} -  d_{\pi (s)\pi(k)}) +  & \\
f_{kr}\cdot (d_{\pi (k)\pi(s)} -  d_{\pi (k)\pi(r)})  +
f_{ks}\cdot (d_{\pi (k)\pi(r)} -  d_{\pi (k)\pi(s)})  &
\end{bmatrix}
\end{displaymath}


\section{estructura del método de búsqueda}

Para el algoritmo BMB se ha seguido el siguiente esquema:

\begin{pythoncode}
Generar 25 soluciones aleatorias
    Para cada una de las soluciones
        Aplicar búsqueda local
        Si la solución dada por la BL mejora la mejor hasta el momento
            Actualizar mejor solución
\end{pythoncode}

Para el algoritmo ILS se ha seguido el siguiente esquema:

\begin{pythoncode}
Generar una solución aleatoria inicial
Aplicarle BL a dicha solución
Si mejora actualizar la mejor solución
Desde 0 .. 23
    mutar la solución actual
    Aplicar BL a la solución mutada
    Si mejora actualizar la mejor solución encontrada
\end{pythoncode}

El operador de mutación consistente en seleccionar una cadena
consecutiva de asignaciones y reasignarlas aleatoriamente, la cadena seleccionada
deberá ser lo suficientemente grande para que el cambio sea significativo, en
este caso t = n/4. El procedimiento seguido es el siguiente:

\begin{pythoncode}
Generar un número aleatorio entre 0 y el tamaño del problema
Calcular cuantos elementos en total habrá que cambiar
Guardar el índice superior
Guardar el índice actual
Para cada uno  de los elementos a cambiar
    generar un nuevo índice entre el índice inferior y el superior
    intercambiar el índice generado con el actual
    actualizar el índice actual a una unidad más
\end{pythoncode}

\section{Algoritmo de comparación, Greedy}

Consiste simplemente en asociar unidades de gran flujo con localizaciones céntricas en la red y viceversa. Para ello se calcula el potencial de flujo y de distancia. A mayor potencial de flujo, más peso tendrá dicha unidad en el intercambio de flujos y, a menor flujo de distancia, más céntrica será la localización. Por tanto el algoritmo selecciona la unidad disponible con mayor potencial de flujo y le asignará la localización de menor potencial de distancia.

\begin{pythoncode}
    Calculo potenciales.
    Inicialzar N a 0.
    para i = 1 hasta el tamaño del problema
        Seleccionar unidad de mayor potencial de flujo
        Seleccionar localización de menor potencial de distancia
        Añadir la localización seleccionada al índice correspondiente a la unidad en N
\end{pythoncode}

\section{Procedimiento considerado}

La implementación se ha realizado en Python, basándose en las explicaciones
de clase y los documentos proporcionados por el profesorado. Para ejecutar el programa:

\begin{bashcode}
$ python QAP.py -d <datos del problema> -a [ils|bmb] -s semilla -v<verbose>
\end{bashcode}

\section{Experimentos y análisis de resultados}

Para esta pŕactica se ha usado todos los casos, salvo los bur. La semilla
usada fue 2704647398.

A continuación se muestra la tabla con los resultados para los distintos algoritmos.

\begin{table}[h]
\centering
    \begin{tabular}{lll}
    \toprule
    Algoritmo               & Desviación & Tiempo \\
    \midrule
    Greedy&81.4492028061&0.0012543466\\
    BMB&5.9915214967&5.6860943106\\
    ILS&5.9334093738&5.2941380077\\
    \bottomrule
    \end{tabular}
    \caption{}
\end{table}
%\end{multicols}


\begin{table}[h!]
\centering
    \begin{tabular}{llll}
    \hline
    \multicolumn{4}{c}{BMB} \\
    \toprule
    Caso               & Coste & Desv & Tiempo \\
    \midrule
Els19&17997928&4.56&0.6709640026\\
Chr20a&2670&21.81&0.4514892101\\
Chr25a&4976&31.09&1.0165770054\\
Nug25&3828&2.24&0.8726320267\\
Tai30a&1890498&3.98&1.870486021\\
Tai30b&650861550&2.16&2.8595149517\\
Esc32a&144&10.77&2.0284779072\\
Kra32&92860&4.69&1.902946949\\
Tai35a&2513334&3.77&2.9234039784\\
Tai35b&287760798&1.57&4.9026348591\\
Tho40&246460&2.47&5.5052878857\\
Tai40a&3266848&4.06&4.134128809\\
Sko42&16070&1.63&6.004180193\\
Sko49&23804&1.79&9.6663868427\\
Tai50a&5138272&4.04&8.1470592022\\
Tai50b&470134271&2.47&12.7287528515\\
Tai60a&7491980&3.97&11.8476259708\\
Lipa90a&363473&0.79&24.8171489239\\

    \bottomrule
    \end{tabular}
    \caption{}
\end{table}


\begin{table}[h!]
\centering
    \begin{tabular}{llll}
    \hline
    \multicolumn{4}{c}{ILS} \\
    \toprule
    Caso               & Coste & Desv & Tiempo \\
    \midrule
    Els19&19612684&13.94&0.3921871185\\
    Chr20a&2870&30.93&0.373308897\\
    Chr25a&4964&30.77&0.8017849922\\
    Nug25&3758&0.37&0.8408501148\\
    Tai30a&1860086&2.31&1.3652291298\\
    Tai30b&654878555&2.79&1.889136076\\
    Esc32a&134&3.08&1.4711718559\\
    Kra32&90570&2.11&1.5701570511\\
    Tai35a&2458870&1.52&2.496491909\\
    Tai35b&294069830&3.80&2.8392338753\\
    Tho40&244544&1.67&3.84405303\\
    Tai40a&3249646&3.51&3.4696700573\\
    Sko42&16012&1.26&4.3035180569\\
    Sko49&23676&1.24&6.9809579849\\
    Tai50a&5082294&2.91&6.7509710789\\
    Tai50b&460949433&0.46&10.516602993\\
    Tai60a&7451524&3.41&11.595993042\\
    Lipa90a&363213&0.72&33.7931668758\\
    \bottomrule
    \end{tabular}
    \caption{}
\end{table}


Ambos algoritmos son bastante parecidos en cuanto a tiempo de ejecución
y resultados. Es posible que éstos resultados estén más influenciados
por la búsqueda local, la cual converge muy rápidamente, y debido a ésto
los resultados son muy similares para ambos casos.

\end{document}
