\documentclass[twoside]{article}


% ------
% Fonts and typesetting settings
%\usepackage[sc]{mathpazo}
\usepackage[T1]{fontenc}
\linespread{1.05} % Palatino needs more space between lines
\usepackage{microtype}


% ------
% Page layout
\usepackage[hmarginratio=1:1,top=32mm,columnsep=20pt]{geometry}
\usepackage[font=it]{caption}
\usepackage{paralist}
\usepackage{multicol}

% ------
% Lettrines
\usepackage{lettrine}


% ------
% Abstract
\usepackage{abstract}
	\renewcommand{\abstractnamefont}{\normalfont\bfseries}
	\renewcommand{\abstracttextfont}{\normalfont\small\itshape}


% ------
% Titling (section/subsection)
\usepackage{titlesec}
\renewcommand\thesection{\Roman{section}}
\titleformat{\section}[block]{\large\scshape\centering}{\thesection.}{1em}{}


% ------
% Header/footer
\usepackage{fancyhdr}
	\pagestyle{fancy}
	\fancyhead{}
	\fancyfoot{}
	\fancyhead[C]{Metaheurísticas $\bullet$ \date{\today} $\bullet$ Práctica 5 - Búsquedas Híbridas}
	\fancyfoot[RO,LE]{\thepage}


% ------
% Clickable URLs (optional)
\usepackage{hyperref}

% ------
% Maketitle metadata
\title{\vspace{-15mm}%
	\fontsize{24pt}{10pt}\selectfont
	\textbf{Práctica 5.a: Búsquedas Híbridas para el Problema de la
    Asignación Cuadrática}
	}
\author{%
	\large
	\textsc{Alejandro Alcalde}\thanks{Template by \href{http://www.howtotex.com}{howtoTeX.com}, \href{http://www.elbauldelprogramador.com}{elbauldelprogramador.com}} \\[2mm]
	\normalsize	Tercer curso Grado en Ingeniería Informática, Granada, grupo de los miércoles a las 17.30 \\
	\normalsize	\href{mailto:algui91@gmail.com}{algui91@gmail.com}	\\
	\normalsize Algoritmos: MA(10,1), MA(10, .1), MA(10, .1best)
	\vspace{-5mm}
	}
\date{}

%%%%%%%%%%%%%%%%%%%%%%%%%%%%%%%%%%%%%%%%%%%%%%%%%%%%%%%%%%%%%%%%%%%%%%%%
% MIS CAMBIOS
%%%%%%%%%%%%%%%%%%%%%%%%%%%%%%%%%%%%%%%%%%%%%%%%%%%%%%%%%%%%%%%%%%%%%%%%

\input{"/home/hkr/Drive/Grado_Ing_Informatica/Global/myConfig"}
\usepackage{booktabs} % Para la tabla
%\let\oldtabular\tabular
%\renewcommand{\tabular}{\footnotesize\oldtabular}
\usepackage{graphicx}
\usepackage{amsmath}                                    % Math



\begin{document}

\maketitle
\thispagestyle{fancy}
\tableofcontents
\newpage

\begin{abstract}
\noindent Memoria de la quinta práctica de la asignatura MetaHeurísticas
del tercer curso del Grado en Ingeniería Informatica de la facultad de Granada.
La practica consiste en la implementación de tres versiones de algoritmos
meméticos usando el genético generacional de la práctica tercera junto
a la búsqueda local de la práctica primera.
\end{abstract}

%\begin{multicols}{2}

\section{Descripción del problema}

\lettrine[nindent=0em,lines=3]{E}l problema de la asignación cuadrática (QAP) es un problema estándar en teoría de localización. En éste se trata de asignar N unidades a una cantidad N de sitios o localizaciones en donde se considera un costo asociado a cada una de las asignaciones. Este costo dependerá de las distancias y flujo entre las unidades, además de un costo adicional por asignar cierta unidad a cierta localización específica. De este modo se buscará que este costo, en función de la distancia y flujo, sea mínimo.

\subsection{Definición matemática}

\begin{displaymath}
_{S\in\prod _N}^{min}\left ( \sum_{i=1}^n \sum_{j=1}^n f_{ij} \cdot d_{S(i)S(j)}  \right )
\end{displaymath}

Donde $\prod _N$ es el conjunto de todas las permutaciones posibles de $N={1,2,\dots,n}$

\subsection{Representación del problema}

En éste problema las soluciones se pueden representar como permutaciones de un conjunto. Si el problema es de tamaño cuatro, por ejemplo, una solución vendría dada por la permutación $N=\{3,2,1,4\}$. Si tomamos los índices de éste conjunto como las unidades, y el valor en dicho índice como localizaciones, la localización 3 estaría asignada a la unidad 1, la localización 2 a la unidad 2 etc.

El objetivo del problema es \textbf{minimizar} la expresión mostrada anteriormente. Ésta será la función objetivo:

\begin{displaymath}
_{S\in\prod _N}^{min}\left ( \sum_{i=1}^n \sum_{j=1}^n f_{ij} \cdot d_{S(i)S(j)}  \right )
\end{displaymath}

Hará falta un mecanismo para \textbf{generar la solución inicial}, en nuestro caso, la permutación $N$ inicial sobre la que lanzar los distintos algoritmos será aleatoria.

De igual modo será necesario un esquema de \textbf{generación de soluciones vecinas}. Para ello, se realizará un intercambio entre dos localizaciones, cambiando sus respectivas unidades por la otra.

\section{Composición de los algoritmos}

\subsection{Esquema de representación}

Como se ha mencionado anteriormente, el esquema elegido para representar
una solución ha sido una permutación del tipo $\{1,2,3,4\}$ en la que los
elementos representan localizaciones y la posición que ocupan las unidades
a las que han sido asignadas.

\subsection{Descripción de la función objetivo}

La función objetivo descrita con su fórmula en apartados anteriores puede
representarse con el siguiente pseudocódigo:

\begin{pythoncode}
i = 0 .. Tamaño del problema
    j = 0 .. Tamaño del problema
        acumular el coste producido al asociar el flujo existente entre
        la unidad i y j y la distancia existente entre las localizaciones
        i y j
\end{pythoncode}

\subsection{Generación de soluciones aleatorias}

El tamaño de la población para esta práctica debe ser de 10, por lo tanto
es necesario generar 10 soluciones iniciales aleatorias. Para así disponer
de una primera población. El mecanismo seguido para obtener la población
inicial, dada una semilla inicial, es el siguiente:

\begin{pythoncode}
desde i=0 hasta el tamaño de la población
    solucion aleatoria = []
    mientras el tamaño de la solución no sea correcto
        generar un número aleatorio entre 0 y el tamaño de la solución
            mientras el número generado ya esté en la solución
                descartarlo y generar uno nuevo
    añadir la solución aleatoria a la población
\end{pythoncode}

Éste proceso se realiza al principio de la ejecución del algoritmo.

\subsection{Algoritmo de búsqueda local empleado}

\begin{pythoncode}
mientras el numero de evaluaciones < (el máx de eval Y improve_flag)
    poner improve_flag a falso
    iterar mientras i < tam problema y no improve_flag
        si en la tabla DLB el elemento i está a 0
            iterar mientras j < tam. Problema y no improve_flag
                delta = factorización(i,j)
                incrementar numero de evaluaciones
                si hubo mejora # delta < 0
                    actualizar el coste actual
                    aplicar el intercambio i,j
                    establecer la DLB para i,j a 0.
                    fijar improve_flag a verdadero
            si no se mejoró # improve_flag a falso
                establecer la DLB para i a 1.
\end{pythoncode}

\subsubsection{Exploración del entorno}

Se emplea una técnica llamada \textit{Don't Look Bits}, que permite focalizar la Búsqueda en una zona del espacio en la que se puede encontrar una solución mejor que la actual.

Con ésta máscara marcamos con 0 todas la unidades inicialmente, indicando que se exploran todos en  orden secuencial. El 1 indica que dicho vecino no va a ser explorado.

\begin{pythoncode}
        DLB = 1..N inicializado a 0
        for i = 1 .. n
            si DLB_i == 0 # se puede mirar el vecino
                for j = 1 .. n
                    vecino = intercambio(i,j)
                    comparar(vecino, solucion actual)
                    Si mejora
                        solucion actual = vecino
                        DLB_i = DLB_j = 0
\end{pythoncode}

\subsubsection{Operador de generación de vecinos}

El operador de vecino simplemente intercambia una posición de la solución
con otra:

\begin{pythoncode}
s = solución
i,j los dos índices a intercambiar
s_i , s_j = s_j , s_i # Se intercambian los valores $s_i$ , $s_j$ = $s_j$ , $s_i$
\end{pythoncode}

\subsubsection{Factorización}

Tras aplicar el operador de vecino, en lugar de calcular de nuevo el coste mediante
la función objetivo, se realiza una factorización que reduce el tiempo de
cálculo considerablemente, ya que evalua únicamente el incremento o decremento
en el coste al aplicar un intercambio entre dos elementos, r y s.

\begin{displaymath}
    \sum_{k \neq r,s}
\begin{bmatrix}
 f_{rk}\cdot (d_{\pi (s)\pi(k)} -  d_{\pi (r)\pi(k)})
+ f_{sk}\cdot (d_{\pi (r)\pi(k)} -  d_{\pi (s)\pi(k)}) +  & \\
f_{kr}\cdot (d_{\pi (k)\pi(s)} -  d_{\pi (k)\pi(r)})  +
f_{ks}\cdot (d_{\pi (k)\pi(r)} -  d_{\pi (k)\pi(s)})  &
\end{bmatrix}
\end{displaymath}

%\subsection{Evaluación}

%Para evitar realizar cálculos innecesarios, durante una generación se marcan
%los individuos que necesitarán ser re-evaluados. Calculando únicamente el coste
%de dichos individuos en el caso de estar marcados para re-evaluación. En
%esta etapa se guarda el mejor individuo de la población.


\section{Estructura del método de búsqueda}

\subsection{MA(10,1)}

El método de búsqueda para esta versión consiste en ejecutar el algoritmo
genético generacional con una población de 10 cromosomas. Cada 10 generaciones
se aplicará la búsqueda local sobre todos los individuos con un máximo de
400 evaluaciones para cada uno. El pseudocódigo es el siguiente:

\begin{pythoncode}
mientras no se cumpla el criterio de parada
    intercambiar la población antigua con la nueva
    seleccionar individuos a formar parte de la nueva generación
    cruzar algunos individuos
    mutar algunos individuos
    reemplazar la población siguiendo esquema elitista.
    evaluar la población nueva

    cada 10 generaciones
        para cada elemento de la población
            aplicar busqueda local con máximo de 400 evaluaciones
            restar las evaluaciones de la BL al criterio de parada
            contador de generaciones a 0
\end{pythoncode}

\subsection{MA(10, 0.1)}

Ésta versión, al igual que la anterior, ejecuta el genético generacional,
cada 10 generaciones aplica la búsqueda local con 400 evaluaciones, pero
ahora se aplica sobre el 10\% de la población. Es decir, $0.1\cdot 10$ (Se
ha hecho así para ahorrar la generación de números aleatorios adicionales)


\begin{pythoncode}
mientras no se cumpla el criterio de parada
    intercambiar la población antigua con la nueva
    seleccionar individuos a formar parte de la nueva generación
    cruzar algunos individuos
    mutar algunos individuos
    reemplazar la población siguiendo esquema elitista.
    evaluar la población nueva

    cada 10 generaciones
        para i = 0 hasta 0.1 * 10
            aplicar BL al elemento i de la población
            restar las evaluaciones de la BL al criterio de parada
            contador de generaciones a 0
\end{pythoncode}

\subsection{MA(10, 0.1Mejor)}

Ésta versión, al igual las anteriores, ejecuta el genético generacional,
cada 10 generaciones aplica la búsqueda local con 400 evaluaciones, pero
ahora se aplica sobre el 10\% de los mejores individuos.

\begin{pythoncode}
mientras no se cumpla el criterio de parada
    intercambiar la población antigua con la nueva
    seleccionar individuos a formar parte de la nueva generación
    cruzar algunos individuos
    mutar algunos individuos
    reemplazar la población siguiendo esquema elitista.
    evaluar la población nueva

    cada 10 generaciones
        obtener los 0.1 * 10 mejores individuos
        guardar los índices de dichos individuos
        para i = cada índice de los mejores individuos
            aplicar BL al elemento i de la población
            restar las evaluaciones de la BL al criterio de parada
            contador de generaciones a 0
\end{pythoncode}

\section{Algoritmo de comparación, Greedy}

Consiste simplemente en asociar unidades de gran flujo con localizaciones céntricas en la red y viceversa. Para ello se calcula el potencial de flujo y de distancia. A mayor potencial de flujo, más peso tendrá dicha unidad en el intercambio de flujos y, a menor flujo de distancia, más céntrica será la localización. Por tanto el algoritmo selecciona la unidad disponible con mayor potencial de flujo y le asignará la localización de menor potencial de distancia.

\begin{pythoncode}
    Calculo potenciales.
    Inicialzar N a 0.
    para i = 1 hasta el tamaño del problema
        Seleccionar unidad de mayor potencial de flujo
        Seleccionar localización de menor potencial de distancia
        Añadir la localización seleccionada al índice correspondiente a la unidad en N
\end{pythoncode}


\section{Procedimiento considerado}

La implementación se ha realizado en Python, basándose en las explicaciones
de clase y los documentos proporcionados por el profesorado. Para ejecutar el programa:

\begin{bashcode}
$ python QAP.py -d <datos del problema> -a \
[ma_10_1_pmx | ma_10_01_pmx | ma_10_01best_pmx] -s semilla -v<verbose>
\end{bashcode}

\section{Experimentos y análisis de resultados}

Para esta pŕactica se han usado todos los casos, salvo los bur. La semilla
usada fue 3327229832

A continuación se muestra la tabla con los resultados para los distintos algoritmos.

\begin{table}[h]
\centering
    \begin{tabular}{lll}
    \toprule
    Algoritmo               & Desviación & Tiempo \\
    \midrule
Greedy&81.4492028061&0.0012273788\\
MA(10,1)&10.8358796412&1.5203776227\\
MA(10,0.1)&9.4251761779&0.0043138133\\
MA(10,0.1mej)&10.6540082225&0.0047982401\\
    \bottomrule
    \end{tabular}
    \caption{}
\end{table}
%\end{multicols}


\begin{table}[h]
\centering
    \begin{tabular}{llll}
    \hline
    \multicolumn{4}{c}{MA(10, 1))} \\
    \toprule
    Caso               & Coste & Desv & Tiempo \\
    \midrule
Els19&17997928&4.56&0.7779898643\\
Chr20a&2706&23.45&0.7962989807\\
Chr25a&5208&37.20&0.6448259354\\
Nug25&3798&1.44&0.935076952\\
Tai30a&1915336&5.35&0.7464430332\\
Tai30b&768457665&20.61&1.3706448078\\
Esc32a&154&18.46&0.8313019276\\
Kra32&94690&6.75&0.9608359337\\
Tai35a&2552422&5.38&1.0090508461\\
Tai35b&327182957&15.48&1.2285130024\\
Tho40&263814&9.69&1.5948040485\\
Tai40a&3333080&6.17&1.4561870098\\
Sko42&16820&6.37&1.6261639595\\
Sko49&24924&6.58&1.8619780541\\
Tai50a&5287240&7.06&1.5881719589\\
Tai50b&511149312&11.40&2.2749328613\\
Tai60a&7779014&7.95&2.8702600002\\
Lipa90a&364706&1.13&4.7933180332\\
    \bottomrule
    \end{tabular}
    \caption{}
\end{table}


\begin{table}[h]
\centering
    \begin{tabular}{llll}
    \hline
    \multicolumn{4}{c}{MA(10, 0.1)} \\
    \toprule
    Caso               & Coste & Desv & Tiempo \\
    \midrule
Els19&18124442&5.30&0.001611948\\
Chr20a&2714&23.81&0.0017259121\\
Chr25a&5664&49.21&0.0017328262\\
Nug25&3788&1.18&0.0025629997\\
Tai30a&1888706&3.88&0.002959013\\
Tai30b&769103949&20.72&0.0029649734\\
Esc32a&150&15.38&0.0038039684\\
Kra32&94920&7.01&0.0034320354\\
Tai35a&2526558&4.32&0.0039310455\\
Tai35b&303551155&7.14&0.0040099621\\
Tho40&246858&2.64&0.0048840046\\
Tai40a&3265130&4.01&0.0042901039\\
Sko42&16500&4.35&0.0051498413\\
Sko49&24164&3.33&0.0058801174\\
Tai50a&5113536&3.54&0.0062999725\\
Tai50b&489069535&6.59&0.005863905\\
Tai60a&7659162&6.29&0.0055539608\\
Lipa90a&364101&0.96&0.0109920502\\
    \bottomrule
    \end{tabular}
    \caption{}
\end{table}


\begin{table}[h]
\centering
    \begin{tabular}{llll}
    \hline
    \multicolumn{4}{c}{MA(10, 0.1mej)} \\
    \toprule
    Caso               & Coste & Desv & Tiempo \\
    \midrule
Els19&18039214&4.80&0.0016949177\\
Chr20a&3084&40.69&0.0018479824\\
Chr25a&6366&67.70&0.0027389526\\
Nug25&3768&0.64&0.0025930405\\
Tai30a&1887820&3.83&0.002820015\\
Tai30b&697468377&9.47&0.0029940605\\
Esc32a&150&15.38&0.0032198429\\
Kra32&92090&3.82&0.0036840439\\
Tai35a&2530044&4.46&0.0039908886\\
Tai35b&305858719&7.96&0.0041918755\\
Tho40&251344&4.50&0.0054271221\\
Tai40a&3329006&6.04&0.0041720867\\
Sko42&16432&3.92&0.0033957958\\
Sko49&24150&3.27&0.0056490898\\
Tai50a&5180742&4.90&0.0076389313\\
Tai50b&478146849&4.21&0.0064709187\\
Tai60a&7577504&5.16&0.0082199574\\
Lipa90a&364257&1.01&0.0156188011\\
    \bottomrule
    \end{tabular}
    \caption{}
\end{table}

\subsection{Observaciones}

El mejor algoritmo globalmente es \pythoninline/MA(10,0.1)/ probablemente
porque tenga un mayor equilibrio entre explotación y exploración. Ya que
solo realizará una explotación mediante búsqueda local a un 10\% de la población,
lo cual deja bastante margen al algoritmo genético para explorar el espacio
de soluciones y no quedar estancado localmente.

En algunas instancias concretas, \pythoninline/MA(10,0.1mej)/ y \pythoninline/MA(10,0.1)/,
obtienen resultados muy cercanos al óptimo. Ejemplos de instancias en las
que ocurre esto, para \pythoninline/MA(10,0.1)/ son \textbf{Nug25} y \textbf{Lipa90a},
con desviaciones de 1.18 y 0.96 respectivamente. Esto se puede deber
a que el algoritmo genético haya realizado una buena exploración y se haya
situado en la región del óptimo, para despues haber explotado ese espacio
con la búsqueda local. Para \pythoninline/MA(10,0.1mej)/ vuelve a destacar
la cercanía con el óptimo de estas dos instancias, en este caso 0.64 para
\textbf{Nug25} y 1.01 para \textbf{Lipa90a}


\end{document}
