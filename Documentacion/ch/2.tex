\section{Instalando PHP-FPM}
En lugar de instalar php5, instalaremos php5-fpm (FastCGI Process Manager),
una implementación alternativa con algunas características adicionales. En
Ubuntu se puede instalar desde repositorios, para debian los agregamos a mano
al \emph{sources.list}:
\begin{bashcode}
deb http://packages.dotdeb.org stable all
deb-src http://packages.dotdeb.org stable all
\end{bashcode}
Es necesario agregar la llave GnuPG, instalamos php5-fpm y lo iniciamos:
\begin{bashcode}
apt-get update
wget http://www.dotdeb.org/dotdeb.gpg
cat dotdeb.gpg | sudo apt-key add -
apt-get install php5-cli php5-suhosin php5-fpm php5-cgi php5-mysql
service php5-fpm start
\end{bashcode}
Ahora probaremos que php funciona bajo nginx, para ello es necesario modificar
ligeramente el archivo nginx.conf, concretamente:
\begin{itemize}
    \item En el bloque \verb!http!  hay que añadir index.php a la directiva
    index, para que quede \emph{index index.php index.html index.htm;}.
    \item Necesitamos crear la comunicación entre nginx y php mediante un
    socket, para ello añadimos lo siguiente en el bloque \verb!http!
        \begin{bashcode}
upstream php {
    server unix://var/run/php-fpm.socket;
}
        \end{bashcode}
    \item Por último, dentro del bloque \verb!server!, añadimos una regla
    que permita manejar los archivos php:
    \begin{bashcode}
location ~ \.php$ {
    include fastcgi_params;
    fastcgi_index index.php;
    fastcgi_param SCRIPT_FILENAME $document_root$fastcgi_script_name;
    fastcgi_pass php;
}
    \end{bashcode}
    \item Una última modificación al archivo \emph{/etc/php5/fpm/pool.d/www.conf}
    y agregamos la línea \emph{listen = /var/run/php-fpm.socket}
\end{itemize}
\subsection{Probando PHP}
Para comprobar que PHP funciona crearemos un fichero simple que mostrará un
mensaje, hemos de colocarlo en \emph{/usr/local/nginx/http/} y asignarle como
grupo y usuario \emph{www-data}:
\begin{bashcode}
echo '<?php echo "Probando que PHP funciona";?>' > /usr/local/nginx/html/index.php
chown www-data:www-data /usr/local/nginx/html/index.php
\end{bashcode}
De nuevo nos dirigimos al \emph{localhost} y deberíamos ver el mensaje, lo
cual indica que se está ejecutando PHP.
