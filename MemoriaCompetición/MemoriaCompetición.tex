\documentclass[twoside]{article}


% ------
% Fonts and typesetting settings
%\usepackage[sc]{mathpazo}
\usepackage[T1]{fontenc}
\linespread{1.05} % Palatino needs more space between lines
\usepackage{microtype}


% ------
% Page layout
\usepackage[hmarginratio=1:1,top=32mm,columnsep=20pt]{geometry}
\usepackage[font=it]{caption}
\usepackage{paralist}
\usepackage{multicol}

% ------
% Lettrines
\usepackage{lettrine}


% ------
% Abstract
\usepackage{abstract}
	\renewcommand{\abstractnamefont}{\normalfont\bfseries}
	\renewcommand{\abstracttextfont}{\normalfont\small\itshape}


% ------
% Titling (section/subsection)
\usepackage{titlesec}
\renewcommand\thesection{\Roman{section}}
\titleformat{\section}[block]{\large\scshape\centering}{\thesection.}{1em}{}


% ------
% Header/footer
\usepackage{fancyhdr}
	\pagestyle{fancy}
	\fancyhead{}
	\fancyfoot{}
	\fancyhead[C]{Metaheurísticas $\bullet$ \date{\today} $\bullet$ Competición}
	\fancyfoot[RO,LE]{\thepage}


% ------
% Clickable URLs (optional)
\usepackage{hyperref}

% ------
% Maketitle metadata
\title{\vspace{-15mm}%
	\fontsize{24pt}{10pt}\selectfont
	\textbf{Competición}
	}
\author{%
	\large
	\textsc{Alejandro Alcalde}\thanks{Template by \href{http://www.howtotex.com}{howtoTeX.com}, \href{http://www.elbauldelprogramador.com}{elbauldelprogramador.com}} \\[2mm]
	\normalsize	Tercer curso Grado en Ingeniería Informática, Granada, grupo de los miércoles a las 17.30 \\
	\normalsize	\href{mailto:algui91@gmail.com}{algui91@gmail.com}	\\
	\normalsize Categorías a participar: Trayectorias, poblacional y mejor sol. Global
	\vspace{-5mm}
	}
\date{}

%%%%%%%%%%%%%%%%%%%%%%%%%%%%%%%%%%%%%%%%%%%%%%%%%%%%%%%%%%%%%%%%%%%%%%%%
% MIS CAMBIOS
%%%%%%%%%%%%%%%%%%%%%%%%%%%%%%%%%%%%%%%%%%%%%%%%%%%%%%%%%%%%%%%%%%%%%%%%

\input{"/home/hkr/Drive/Grado_Ing_Informatica/Global/myConfig"}
\usepackage{booktabs} % Para la tabla
%\let\oldtabular\tabular
%\renewcommand{\tabular}{\footnotesize\oldtabular}
\usepackage{graphicx}
\usepackage{amsmath}                                    % Math



\begin{document}

\maketitle
\thispagestyle{fancy}
\tableofcontents
\newpage

%\begin{multicols}{2}

\section{Modificaciones realizadas sobre los algoritmos de prácticas}

La única modificación notable ha sido realizada al Híbrido, en el cual
se ha implementado el operador de cruce Swap Path Crossover (SPX). En lugar
de escoger el punto de corte aleatorio, se ha considerado comenzar siempre
por el primer elemento.

\section{Manual de ejecución}

La implementación se ha realizado en Python. Es necesario tener instalado
el módulo Numpy.

\begin{bashcode}
$ pip install numpy
\end{bashcode}

Tras esto, el programa se puede ejecutar con los siguientes argumentos:

\begin{bashcode}
$ python QAP.py -d <datos del problema> -a \
[CompPob | comptray] -s 12345678 -v<verbose>
\end{bashcode}

Los directorios con las instancias y soluciones se encuentran en
\pythoninline/Competición/ y \pythoninline/CompeticiónILS/.

Para comprobar que los resultados son correctos, se ha incluido un
programa llamado \pythoninline/test_results.py/ que comprueba si las
soluciones escritas en los ficheros .sln son en realidad correctas, para
ello se les aplica la función objetivo de nuevo. Se usa como sigue:

\begin{pythoncode}
python test_results.py -d Competición
python test_results.py -d CompeticiónILS
\end{pythoncode}


\section{Experimentos y análisis de resultados}

A continuación se muestra la tabla con los resultados para los distintos algoritmos.

\begin{table}[h]
\centering
    \begin{tabular}{lll}
    \toprule
    Algoritmo               & Desviación & Tiempo \\
    \midrule
GA-LS&4.53&0.05\\
ILS&3.76&12.28\\
    \bottomrule
    \end{tabular}
    \caption{}
\end{table}
%\end{multicols}


\begin{table}[h]
\centering
    \begin{tabular}{llll}
    \hline
    \multicolumn{4}{c}{GA-LS} \\
    \toprule
    Caso               & Coste & Desv & Tiempo \\
    \midrule
Esc64a&116&0.00&0.03\\
Esc128&66&3.13&0.12\\
Lipa90b&15340052&22.81&0.06\\
Sko64&50028&3.15&0.04\\
Sko72&68964&4.09&0.04\\
Sko81&94922&4.31&0.06\\
Sko90 &119310&3.27&0.06\\
Sko100a&158918&4.55&0.06\\
Tai64c&1858710&0.15&0.04\\
Tai80a&14238716&5.48&0.04\\
    \bottomrule
    \end{tabular}
    \caption{}
\end{table}


\begin{table}[h]
\centering
    \begin{tabular}{llll}
    \hline
    \multicolumn{4}{c}{ILS} \\
    \toprule
    Caso               & Coste & Desv & Tiempo \\
    \midrule
116&0.00&8.89\\
66&3.13&41.31\\
15297919&22.48&14.26\\
49134&1.31&8.86\\
67868&2.43&10.54\\
93432&2.67&11.47\\
118526&2.59&13.33\\
156602&3.03&14.14\\
1857646&0.09&9.65\\
13973342&3.51&13.58\\

    \bottomrule
    \end{tabular}
    \caption{}
\end{table}

\end{document}
